\documentclass[a4paper]{article}

\usepackage[nocomments]{latexml}
\usepackage{latexmlleeds}

\usepackage{hyperref}
\usepackage[dvipsnames]{xcolor}

\usepackage{listings}
\lstset{basicstyle=\small\ttfamily,
  keywordstyle=\color{blue}\bfseries,
  ndkeywordstyle=\color{ForestGreen}\bfseries,
  stringstyle=\color{red},
  commentstyle=\color{ForestGreen}
}

\lstdefinelanguage{CSS}{sensitive,
  keywords={accelerator,azimuth,background,background-attachment,background-color,background-image,background-position,background-position-x,background-position-y,background-repeat,behavior,border,border-bottom,border-bottom-color,border-bottom-style,border-bottom-width,border-collapse,border-color,border-left,border-left-color,border-left-style,border-left-width,border-right,border-right-color,border-right-style,border-right-width,border-spacing,border-style,border-top,border-top-color,border-top-style,border-top-width,border-width,bottom,caption-side,clear,clip,color,content,counter-increment,counter-reset,cue,cue-after,cue-before,cursor,direction,display,elevation,empty-cells,filter,float,font,font-family,font-size,font-size-adjust,font-stretch,font-style,font-variant,font-weight,height,ime-mode,include-source,layer-background-color,layer-background-image,layout-flow,layout-grid,layout-grid-char,layout-grid-char-spacing,layout-grid-line,layout-grid-mode,layout-grid-type,left,letter-spacing,line-break,line-height,list-style,list-style-image,list-style-position,list-style-type,margin,margin-bottom,margin-left,margin-right,margin-top,marker-offset,marks,max-height,max-width,min-height,min-width,-moz-binding,-moz-border-radius,-moz-border-radius-topleft,-moz-border-radius-topright,-moz-border-radius-bottomright,-moz-border-radius-bottomleft,-moz-border-top-colors,-moz-border-right-colors,-moz-border-bottom-colors,-moz-border-left-colors,-moz-opacity,-moz-outline,-moz-outline-color,-moz-outline-style,-moz-outline-width,-moz-user-focus,-moz-user-input,-moz-user-modify,-moz-user-select,orphans,outline,outline-color,outline-style,outline-width,overflow,overflow-X,overflow-Y,padding,padding-bottom,padding-left,padding-right,padding-top,page,page-break-after,page-break-before,page-break-inside,pause,pause-after,pause-before,pitch,pitch-range,play-during,position,quotes,-replace,richness,right,ruby-align,ruby-overhang,ruby-position,-set-link-source,size,speak,speak-header,speak-numeral,speak-punctuation,speech-rate,stress,scrollbar-arrow-color,scrollbar-base-color,scrollbar-dark-shadow-color,scrollbar-face-color,scrollbar-highlight-color,scrollbar-shadow-color,scrollbar-3d-light-color,scrollbar-track-color,table-layout,text-align,text-align-last,text-decoration,text-indent,text-justify,text-overflow,text-shadow,text-transform,text-autospace,text-kashida-space,text-underline-position,top,unicode-bidi,-use-link-source,vertical-align,visibility,voice-family,volume,white-space,widows,width,word-break,word-spacing,word-wrap,writing-mode,z-index,zoom},
  morecomment=[s]{/*}{*/},
  alsoletter={-}}

\usepackage{amsthm}
\theoremstyle{definition}
\newtheorem{exa}{Example}[subsection]

\title{LaTeXML Leeds tweaks}
\author{Vincenzo Mantova}
\date{24 August 2020}

\begin{document}

\maketitle

\tableofcontents

\section{How to compile}
First, make sure that the required files are next to the \verb|.tex| source.
\begin{itemize}
  \item \verb|latexmlleeds.sty|
  \item \verb|latexmlleeds.sty.ltxml|
  \item \verb|latexmlleeds.css|
  \item \verb|latexmlleeds-html5.xsl|
  \item \verb|latexml.sty| if your \TeX{} installation does not pick it up automatically
\end{itemize}

\begin{lstlisting}[language=bash,caption={Generate the PDF}]
  latexmk LaTeXML-Leeds
\end{lstlisting}
\begin{lstlisting}[language=bash,caption={Generate the EPUB}]
  latexmlc --splitat=chapter --css=latexmlleeds.css \
    --css=latexmlleeds-video.css --destination=LaTeXML-Leeds.epub \
    LaTeXML-Leeds.tex
\end{lstlisting}
\begin{lstlisting}[language=bash,caption={Generate the HTML}]
  latexmlc \
    --javascript="https://cdn.jsdelivr.net/npm/mathjax@3/es5/tex-mml-chtml.js" \
    --mathtex --stylesheet=latexmlleeds-html5.xsl \
    --css=latexmlleeds.css  --css=latexmlleeds-video.css \
    --destination=LaTeXML-Leeds.html LaTeXML-Leeds.tex
\end{lstlisting}
\begin{lstlisting}[language=bash,caption={Generate the HTML in two steps}]
  latexml --destination=LaTeXML-Leeds.xml LaTeXML-Leeds.tex
  latexmlpost \
    --javascript="https://cdn.jsdelivr.net/npm/mathjax@3/es5/tex-mml-chtml.js"\
    --mathtex --stylesheet=latexmlleeds-html5.xsl \
    --css=latexmlleeds.css --css=latexmlleeds-video.css \
    --destination=LaTeXML-Leeds.html LaTeXML-Leeds.xml
\end{lstlisting}

\section{Documentation}

\subsection{The \texttt{latexml} package}
First, ensure you import the packages in your preamble:
\begin{lstlisting}[language=TeX,caption={Import packages in the preamble}]
  \usepackage[nocomments]{latexml}
  \usepackage{latexmlleeds}
\end{lstlisting}

The package \verb|latexml| takes some options that will be passed to the \LaTeXML{} engine, such as \verb|nocomment| (that strips the \verb|%| comments away) or \verb|mathparsespeculate|. Moreover, it offers a variety of commands. The most important one:
\begin{lstlisting}[language=TeX]
  \iflatexml
    % code only executed by latexml
  \else
    % code only executed by the other engines
  \fi
\end{lstlisting}

\begin{exa}
  Code:
  \begin{lstlisting}[language=TeX]
    \usepackage{latexml}
    ...
    \begin{document}
    ...
    \iflatexml Great, you used \LaTeXML{} successfully!
    \else This is a boring PDF.\fi
  \end{lstlisting}
  Output:
  \begin{quote}
    \iflatexml Great, you used \LaTeXML{} successfully!
    \else This is a boring PDF.\fi
  \end{quote}
\end{exa}

For reference, \verb|latexml| also defines the following, although you can work very well without them.
\begin{itemize}
  \item \verb|\lxAddClass{class}| and \verb|\lxWithClass{class}{content}| to add CSS classes to the output;
  \item \verb|\lxBeginTableHead|, \verb|\lxEndTableHead| and variations to mark table headers and footers (read the \verb|latexml.sty| source for how to use them);
  \item \verb|\lxFcn{code}|, \verb|\lxID{code}|, \verb|\lxPunct{code}| to aid \LaTeXML{} understand the meaning of mathematical symbols (\LaTeXML{} recognises the meaning on its own, but every once in a while, there will be a symbol that is just too ambiguous -- just keep an eye on the warnings during compilation);
  \item \verb|\lxContextTOC|, \verb|\lxNavbar{arg}|, \verb|\lxHeader{arg}|, \verb|\lxFooter{arg}| to customise the HTML pages (I have yet to figure out how they work);
  \item and a few other commands.
\end{itemize}

\subsection{The \texttt{latexmlleeds} package}
\verb|latexmlleeds| defines \verb|\lxHTML{html}|, which outputs the \verb|html| argument directly as \HTML{} code. The \verb|html| is only processed by \LaTeXML{} and ignored by the other engines. The \LaTeXML{} processing is done by \verb|latexmlleeds.sty.lxtml|, which is a short Perl file.
\begin{exa}
  Code:
  \begin{lstlisting}[language=TeX]
    This sentence appears in all versions.
    \lxHTML{<span>But this part is
      <strong>only in the HTML outputs</strong>.</span>}
  \end{lstlisting}
  Output:
  \begin{quote}
    This sentence appears in all versions.
    \lxHTML{<span>But this part is
    <strong>only in the HTML outputs</strong>.</span>}
  \end{quote}
\end{exa}

Some notes:
\begin{itemize}
  \item The \HTML{} must be written in ``\XML{} serialisation'', hence all tags must be closed (e.g.\ \verb|<br>| must be written as \verb|<br/>|) and all attributes must have a value (e.g.\ \verb|allowfullscreen| must be written \verb|allowfullscreen=""|).
  \item The content must be written inside \HTML{} elements. Any text not contained in an element will be dropped. If in doubt, put \verb|<span>...</span>| around your \HTML{}.
  \item The \HTML{} is still interpreted as \LaTeX{} code, hence it must be escaped appropriately. For instance, every \verb|{| must be replaced by \verb|\{|; every \verb|&| must be replaced by \verb|\&| (and probably be written as \verb|\&amp;| -- you must write correct \HTML{}!).
\end{itemize}

\begin{exa}[Embed Stream videos]\label{exa:embed-stream}
  This is only a proof of concept: it embeds the video directly into \HTML{} and EPUB, and provides a link in all contexts (note that embedded videos may not work everywhere, especially with EPUB, so always include the link). The code is accompanied by some extra \CSS{} to make the video size responsive, however you must customize it to match the resolution of your videos (mine is $1280 \times 720$).
  \begin{lstlisting}[language=TeX]
    % preamble
    \newcommand{\includestream}[2]{
    \lxHTML{<div class="ltx\_leeds\_video">
        <iframe 
          src="https://web.microsoftstream.com/embed/video/#1?autoplay=false\&amp;showinfo=true"
          allowfullscreen="">
        </iframe>
      </div>}
      Watch video \href{https://web.microsoftstream.com/video/#1}{#2}
    }
    % document
    \includestream{ba6b8866-df29-4dea-a47e-13decc5cd409}
    {Mock recording for Models and Sets}
  \end{lstlisting}

  Output:
  \begin{quote}
    \newcommand{\includestream}[2]{
      \lxHTML{<div class="ltx\_leeds\_video">
          <iframe 
            src="https://web.microsoftstream.com/embed/video/#1?autoplay=false\&amp;showinfo=true"
            allowfullscreen="">
          </iframe>
        </div>}
      Watch video \href{https://web.microsoftstream.com/video/#1}{#2}
    }
    \includestream{ba6b8866-df29-4dea-a47e-13decc5cd409}
    {Mock recording for Models and Sets}
  \end{quote}
\end{exa}

\subsection{The \texttt{latexmlleeds-html5.xsl} customisations}
The \verb|latexmlleeds.xsl| \XSLT{} stylesheet does only three things:
\begin{itemize}
  \item make sure you output HTML5 (which would be the default, anyway);
  \item it makes \HTML{} lists more accessible by ignoring custom enumerations (i.e., the \verb|(a)| in \verb|\item[(a)]|; you should modify your \CSS{} to do custom enumerations in \HTML{});
  \item it adds modern HTML5 tags to make the output more mobile friendly, and removes the outdated \verb|Content-type| meta tag.
\end{itemize}
The \XSLT{} may be updated in the future to improve a few aspects of the resulting \HTML{}; for instance, we might discover that offloading the maths parsing entirely to MathJax is better overall, and this can only be done by tweaking the \XSLT{}.

\subsection{The \texttt{latexmlleeds.css} customisations}
For now, \verb|latexmlleeds.css|:
\begin{itemize}
  \item sets the font to sans serif (trying to pick the best font available);
  \item reduces the margins to a minimum;
  \item sets the maximum text width to about 80 characters;
  \item reduces the margins on the lists;
  \item removes indentation;
  \item implements responsive styling for embedded videos (see \autoref{exa:embed-stream}).
\end{itemize}
This styling has only been tested with the \verb|article| class and on Edge and Chrome, so further tweaks may be needed.

\section{Good practices}
\begin{itemize}
  \item follow Martin's recommendations about fonts, tables, margins, and so on;
  \item use semantic \LaTeX{} commands as much as possible (\verb|\chapter|, \verb|\section|, \verb|\newtheorem|).
\end{itemize}

\end{document}

\section{Adjusting the .tex source}

The sections below are a log of the issues I have found with my files, and how I fixed them. Most of the changes happen in the preamble (or in my case, in \verb|notes.cls.tex|), and most are caused by my unorthodox usage of \LaTeX{} -- I marked as ``skippable'' the section that are unlikely to be interesting.

The trickiest part is working with TikZ. However, the set up is fully automated: once you make some minor changes to your \verb|.tex| files, the commands to run are always the same (hence they are easily scriptable). The attached \verb|Makefile| is programmed to run all the relevant commands when running \verb|make|.

As general rule, it is a good idea to add \verb|\usepackage{latexml}| to your preamble. You will then be able to use
\begin{lstlisting}[style=TeX]
  \iflatexml
    % Code that runs only under LaTeXML
  \else
    % Code that runs only with PDFLaTeX (or your favourite engine)
  \fi
\end{lstlisting}
which is immensely convenient to deal with issues that are best dealt with by using different code for \LaTeXML{} (see the handling of TikZ below). If the package \verb|latexml.sty| gives you an error because your system does not have it, you can copy it from the folder containing your \LaTeXML{} installation into the folder containing your \verb|.tex| file (or just from this document -- the file included here is taken from \LaTeXML{} 0.8.4).

\begin{exa}
  Code:
  \begin{lstlisting}[style=TeX]
    \usepackage{latexml}
    ...
    \begin{document}
    ...
    \iflatexml Great, you used \LaTeXML{} successfully!
    \else This is a boring PDF.\fi
  \end{lstlisting}
  Output:
  \begin{quote}
    \iflatexml Great, you used \LaTeXML{} successfully!
    \else This is a boring PDF.\fi
  \end{quote}
\end{exa}

\subsection{Recognise custom classes and packages (skippable)}

I use the custom class \verb|notes.cls| and the custom package \verb|unimathsym.sty|. By default, \LaTeXML{} ignores classes and packages for which it does not have ``bindings''.

\begin{sol}[quick and dirty]
  Add \verb|--includestyles| to the \verb|latexml| command. However, \LaTeXML{} will then parse \emph{every other unrecognised package}, and this is likely to cause other errors.
\end{sol}

\begin{sol}[safe and still rather quick]
  Move your custom code into one or more \verb|.tex| files and use \verb|\input| like most normal people. Make sure to use your \verb|\makeatletter| and \verb|\makeatother| appropriately. I ended up with this approach, as it is more stable.
\end{sol}

\subsection{Make unimathsym work (skippable)}
My \verb|unimathsym| package uses the \verb|\DeclareUnicodeCharacter| macro from \verb|inputenc|. It turns out that the \LaTeXML{} binding for \verb|inputenc| does not have that macro.

\begin{sol}
  Copy the definition of \verb|\DeclareUnicodeCharacter| and its dependencies directly from \verb|inputenc|, gated behind \verb|\iflatexml ...\fi| to avoid spoiling the normal workflow with double definitions.
\end{sol}

\begin{exa}
  You should see the Weierstrass $℘$ rendered as an equation.
\end{exa}

\textbf{To do.} Streamline \verb|unimathsym| to work only with the characters I actually need, as it is so big that it slow down \LaTeXML{} considerably.

\subsection{Make TikZ work}
TikZ, or rather the underlying PGF library, is too clever for \LaTeXML{}, at least version 0.8.4. Those issues are supposed to be fixed in 0.8.5, which has not been released yet.

The approach below may work well for a variety of situations where a package produces an image, or run a piece of code, or else, provided you have the means to extract the outputs into separate files. I recommend always producing SVG images because they are scalable: you will be able to zoom in without losing sharpness, and the images will work well on high resolution displays.

\begin{sol}
  The workaround is to use TikZ to produce external images during \emph{normal} compilation. We then convert the images from PDF to SVG, and we use \verb|\iflatexml| include those images in place of the original \verb|tikzpicture|'s. There is a \href{https://github.com/brucemiller/LaTeXML/issues/945}{clever way} of doing this that is almost completely automated.

  \textbf{Step 1.} Include the following code in the preamble:
  \begin{lstlisting}[style=TeX]
    \iflatexml
      \usepackage{graphicx}
      % delete graphicx was already imported
      \DeclareGraphicsExtensions{%
        .svg,.SVG,%
        .png,.PNG,%
        .pdf,.PDF,%
        .eps,.EPS,%
        .jpg,.mps,.jpeg,.jbig2,.jb2,.JPG,.JPEG,.JBIG2,.JB2}
      \newcounter{tikzpicturecounter}
      \newcommand{\includetikzexternalized}{
        \includegraphics[scale=1.5]{images/TikZ-\thetikzpicturecounter}
        \stepcounter{tikzpicturecounter}
      }
    \else
      \usepackage{tikz}
      \usetikzlibrary{arrows}
      \usetikzlibrary{external}
      \tikzexternalize
      \tikzsetfigurename{images/TikZ-}
    \fi
  \end{lstlisting}
  \textbf{Step 2.} Wrap every \verb|\begin{tikzpicture} ... \end{tikzpicture}| within
  \begin{lstlisting}[style=TeX]
    \iflatexml
    \includetikzexternalized
    \else
    \begin{tikzpicture}
      ...
    \end{tikzpicture}
    \fi
  \end{lstlisting}
  \textbf{Step 3.} Create the folder \verb|images| and run \verb|pdflatex| with the option \verb|-shell-escape|. This will generate the images.

  \noindent\textbf{Step 4.} Convert the images to SVG using \verb|pdf2svg|:
  \begin{lstlisting}[language=bash]
    for file in images/TikZ-*.pdf
      do pdf2svg $file ${file%.pdf}.svg
    done
  \end{lstlisting}
  Now, running \LaTeXML{} will pick up the images in the folder \verb|images| instead of trying to run TikZ.

  \noindent\textbf{Step 5 (optional).} \LaTeXML{} throws out warnings about not being able to determine the image sizes. But it also offers the solution: install one of \verb|Graphics::Magick| or \verb|Image::Magick|. Thus, install ImageMagick or GraphicsMagick, and it will stop complaining.

  \noindent\textbf{To do.} Figure out how to add alt text to the images, which may be required for accessibility.
\end{sol}

\subsection{Drop mdframed, prettyref, cleveref}

The packages \verb|mdframed|, \verb|prettyref| are not supported by \LaTeXML{}, and my use of \verb|cleveref| is incompatible. I use:
\begin{itemize}
  \item \verb|mbframed| to frame definitions, theorems, axioms. It was very popular with the students.
  \item \verb|prettyref| to generate references that automatically include ``Theorem'' in front of the number.
  \item \verb|cleveref| to generate references of the form ``Table 5.4.1 on this page'' or ``Table 5.4.2 on the next page''. This does not make sense for \HTML{} and EPUB since they are not paged, hence \LaTeXML{} crashes.
\end{itemize}

\begin{sol}[quick and reasonable]
  When processing with \LaTeXML{} (hence all gated behind \verb|\iflatexml|):
  \begin{itemize}
    \item \verb|mdframed|: revert to \verb|thm|, \verb|defn|, \verb|axiom| to the normal \verb|\newtheorem| (see \verb|notes.cls.tex|);
    \item \verb|prettyref|: redefine \verb|\prettyref| to coincide with \verb|\autoref| and nullify \verb|\newrefformat| when processing with \LaTeXML{};
    \item \verb|cleveref|: redefine \verb|\Vref| to \verb|\prettyref|
      \begin{lstlisting}[style=TeX]
        \iflatexml
        \let\Vref\prettyref
        \else
        \RequirePackage{cleveref}
        \fi
      \end{lstlisting}
  \end{itemize}
  I have yet to verify that all references appear properly with \verb|\autoref|.
\end{sol}

\begin{sol}[for the future]
  Investigate how to produced framed theorems in the \HTML{}/EPUB versions, and check if I can get rid of \verb|\prettyref| completely by using \verb|\autoref| (the functionality is slightly different!).
\end{sol}

\subsection{The \textbackslash{}not@math@alphabet mystery (skippable)}
My file causes the following error when using \verb|--includestyles|.
\begin{quote}
  \verb|Error:undefined:\not@math@alphabet The token T_CS[\not@math@alphabet] is not defined.|
\end{quote}

\begin{sol}[proper]
  Do not use \verb|--includestyles|! As I said, it is risky.
\end{sol}

\begin{sol}[if you insist]
  In \LaTeX{} (more precisely, in \verb|tex/latex/base/latex209.def|), \verb|\not@math@alphabet| is defined via
  \begin{lstlisting}[style=TeX]
    \let\not@math@alphabet\@gobbletwo
  \end{lstlisting}
  so I just copy that code anywhere before \verb|\begin{document}|.
\end{sol}

\subsection{Underbrace issues}
Something weird is going on with \verb|\underbrace|. This seems to be caused by the option \verb|--noparse|, which I needed when using \verb|--includestyles| because \LaTeXML{} would get stuck in an infinite loop over some formula.

\begin{sol}
  Do not use \verb|--includestyles| and \verb|--noparse|! As a safety measure, I also replaced the Unicode characters for with the actual command \verb|\underbrace|.
\end{sol}

\section{Beautifying the result}

\subsection{Split sections or chapters into different files}

Add \verb|--splitat=section| (or chapter) when running \verb|latexmlpost|. This is important to generate a navigable output and to make equations appear quickly. It is also important for EPUBs that contiguous blocks of pages are not too long, hence split at least by chapter. See \href{https://dlmf.nist.gov/LaTeXML/manual/usage/splitting/}{§ 2.3 Splitting the Output} of the \LaTeXML{} documentation for details about the various options, including splitting the bibliography and file naming (for instance, you can split sections into separate folders each with an \verb|index.html| file, which makes the resulting URLs nicer).

\textbf{Advice.} Compile your output into a subfolder by using \verb|--destination=subfolder/mynotes.html|, since it will generate many files.

\subsection{Make lists more accessible}

\LaTeXML{} goes out of its way to preserve the appropriate \verb|itemize| bullets and \verb|itemize| numberings. However, this creates weird \HTML{} that may be slightly confusing to screen readers.

\begin{sol}
  Add \verb|--xsltparameter=SIMPLIFY_HTML:true| when calling \verb|latexmlpost|, or add
  \begin{lstlisting}[language=XML]
    <xsl:param name="SIMPLIFY_HTML">true</xsl:param>
  \end{lstlisting}
  to your custom \XSLT{} stylesheet (see next subsection).
\end{sol}

\textbf{Price to pay.} This will break any customisation to your lists (including starting \verb|enumerate|'s from a number different than one). You can recover some of the customization by tweaking the \CSS{} linked to your \HTML{} files.

\subsection{Make the output more mobile friendly}

The \LaTeXML{} is not modern enough to scale properly on mobile devices. One needs to add an appropriate \HTML{} tag to the header of the \HTML{} files. This can be done by tweaking the default \XSLT{} stylesheet.

\textbf{To do.} To make the output \emph{fully} mobile friendly, one needs to tweak the \CSS{} a bit more to remove excessive margin, change the layout of navigation header and footer, and possibly some other tricks. However, the change below does 90\% of the work.

\textbf{Step 1.} Create the custom \XSLT{} stylesheet. For the starting point, see \verb|LaTeXML-html5.xsl|, which is already very short. The stylesheet below will add the appropriate viewport tag; it will also add the standard charset tag up top, and remove the Content-type tag, as recommended by the latest \HTML{}5 standards.

\begin{lstlisting}[language=XML]
  <?xml version="1.0" encoding="utf-8"?>
  <xsl:stylesheet
      version   = "1.0"
      xmlns:xsl = "http://www.w3.org/1999/XSL/Transform"
      xmlns:ltx = "http://dlmf.nist.gov/LaTeXML"
      exclude-result-prefixes="ltx">

    <!-- Include the standard LaTeXML \HTML{}5 stylesheet -->
    <xsl:import href="urn:x-LaTeXML:XSLT:LaTeXML-html5.xsl"/>

    <!-- attach a few extra tags at the beginning of <head> -->
    <xsl:template match="/" mode="head-begin">
      <meta charset="UTF-8" />
      <meta name="viewport" content="width=device-width, initial-scale=1"/>
    </xsl:template>

    <!-- modernise the output by removing the Content-type meta tag -->
    <xsl:template match="/" mode="head-content-type"/>

  </xsl:stylesheet>
\end{lstlisting}

\textbf{Step 2.} Call \verb|latexmlpost| with the additional option \verb|--stylesheet=latexmlleeds-html5.xsl| (where \verb|latexmlleeds-html5.xsl| is the name of your custom \XSLT{} stylesheet).

\subsection{Adding custom \HTML{}, such as embedding videos}
I have written a tiny \LaTeX{} package, with corresponding \LaTeXML{} binding, that you can use to write \HTML{} that is sent directly to \LaTeXML{} using the new \verb|\lxHTML{}| command.
\begin{exa}
  Copy the files \verb|latexmlleeds.sty| and \verb|latexmlleeds.sty.ltxml| into your \TeX{} folder, and write:
  \begin{lstlisting}[style=TeX]
    \usepackage{latexmlleeds}
    ...
    \begin{document}
    ...
    This sentence appears in all versions.
    \lxHTML{<span>But this part is <strong>only in the HTML outputs</strong>.</span>}
  \end{lstlisting}
  Output:
  \begin{quote}
    This sentence appears in all versions. \lxHTML{<span>But this part appears <strong>only in the \HTML{} outputs</strong>.</span>}
  \end{quote}
\end{exa}

Some notes:
\begin{itemize}
  \item The \HTML{} must be written in ``XML serialisation'', hence all tags must be closed (e.g.\ \verb|<br>| must be written as \verb|<br/>|) and all attributes must have a value (e.g.\ \verb|allowfullscreen| must be written \verb|allowfullscreen=""|).
  \item The content must be written inside \HTML{} elements. Any text not contained in an element will be dropped. If in doubt, put \verb|<span>...</span>| around your \HTML{}.
  \item The \HTML{} is still interpreted as \LaTeX{} code, hence it must be escaped appropriately. For instance, every \verb|{| must be replaced by \verb|\{|; every \verb|&| must be replaced by \verb|\&| (and probably be written as \verb|\&amp;| -- write correct \HTML{}!).
\end{itemize}

The intended use case is to embed Stream videos inside the lecture notes. 

\begin{exa}[Proof of concept]
  In the example below, the link is included in \emph{every} format, not just PDF, so that it still works in EPUB readers that ignore \verb|<iframe>|. Note how the \HTML{} code is still \LaTeX{} code.
  \begin{lstlisting}[style=TeX]
    \lxHTML{<div style="position: relative; width: 100\%; max-width: 1280px;
      max-height: 720px; height: 0; padding-top: min(720px,56.25\%); overflow: hidden;">
      <iframe style="position: absolute; top: 0; left: 0; width: 100\%;
        height: 100\%; border: none;" width="640" height="360"
      src="https://web.microsoftstream.com/embed/video/#1?autoplay=false\&amp;showinfo=true"
      allowfullscreen=""></iframe></div>}
      \href{https://web.microsoftstream.com/video/#1}{Watch video ``#2''}
    }
    ...
    \includestream{ba6b8866-df29-4dea-a47e-13decc5cd409}{Mock recording for Models and Sets}
  \end{lstlisting}

  Output:
  \begin{quote}
    \newcommand{\includestream}[2]{
      \lxHTML{<div style="position: relative; width: 100\%; max-width: 1280px;
        max-height: 720px; height: 0; padding-top: min(720px,56.25\%); overflow: hidden;">
      <iframe style="position: absolute; top: 0; left: 0; width: 100\%;
        height: 100\%; border: none;" width="640" height="360"
      src="https://web.microsoftstream.com/embed/video/#1?autoplay=false\&amp;showinfo=true"
      allowfullscreen=""></iframe></div>}
      \href{https://web.microsoftstream.com/video/#1}{Watch video ``#2''}
    }
    \includestream{ba6b8866-df29-4dea-a47e-13decc5cd409}{Mock recording for Models and Sets}
  \end{quote}

  Some notes:
  \begin{enumerate}
    \item the \verb|<iframe>| is followed by a link in every format; this makes the output more resilient (for instance, many EPUB readers will omit the \verb|<iframe>|);
    \item the \HTML{} code is still \LaTeX{} code as well: it can take arguments such as \verb|#1|, and \verb|&| must be escaped appropriately;
    \item there is \CSS{} trickery to make the embedded videos responsive (assuming the resolution is $1280×720$, hence with ratio $720/1280 = 56.25\%$); the correct solution is to add a suitable class and provide additional \CSS{} separately.
  \end{enumerate}
\end{exa}

Note that the two files contain very little code. Stripped of comments and metadata, they look like the following:
\begin{lstlisting}[style=TeX]
  % latexmlleeds.sty
  \RequirePackage[T1]{fontenc}
  \newcommand{\lxHTML}[1]{}
\end{lstlisting}
\begin{lstlisting}[language=Perl]
  # latexmlleeds.sty.ltxml
  use LaTeXML::Package;
  use XML::LibXML;
  my $parser = XML::LibXML->new();
  RequirePackage('fontenc', options => ['T1']);
  DefMacro('\lxHTML{}', '');
  DefConstructor('\lxHTML{}', sub {
    my ($document, $arg) = @_;
    my $html = $parser->parse_balanced_chunk('<span xmlns="http://www.w3.org/1999/xhtml">' . $arg->toString() . '</span>');
    my @elems = $html->firstChild->findnodes('*');
    my $node = $document->openElement('ltx:rawhtml');
    map { $document->appendClone($node, $_) } @elems;
    $document->closeElement('ltx:rawhtml');
  });
  1;
\end{lstlisting}

\subsection{Tweaking the appearance}
The default \LaTeXML{} output mimics the PDF style too closely. I tweaked the appearance by writing some additional \CSS{} in \verb|latexmlleeds.css|:
\begin{lstlisting}[language=CSS]
  /* Set font to sans serif */
  body {
    font-family: -apple-system, BlinkMacSystemFont, 'Segoe UI', Roboto, Oxygen, Ubuntu, Cantarell, 'Open Sans', 'Helvetica Neue', Helvetica, Arial, sans-serif, "Apple Color Emoji", "Segoe UI Emoji", "Segoe UI Symbol";
  }

  /* Reduce margins to minimum and set max text width to about 80 characters */
  .ltx_page_main {
    padding: 5px;
    margin: 0 auto;
    max-width: 80ex;
  }

  /* Reduce margins on lists */
  .ltx_itemize, .ltx_enumerate {
    margin-left: inherit;
  }

  /* Remove indentation on paragraphs */
  .ltx_para > .ltx_p:first-child {
    text-indent: 0em !important;
  }
\end{lstlisting}

\end{document}