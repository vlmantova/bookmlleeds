\documentclass[a4paper,british]{article}

\usepackage[T1]{fontenc}
\usepackage[british]{babel}

\usepackage[pdfusetitle,colorlinks]{hyperref}
\usepackage[dvipsnames]{xcolor}

\usepackage{lmodern}

\usepackage{bookml/bookml}
\bmlAltFormat{bookmlleeds.pdf}{PDF (serif)}
\bmlAltFormat{bookmlleeds-sans.pdf}{PDF (sans serif)}
\bmlAltFormat{bookmlleeds-sans-large.pdf}{PDF (sans, large)}

\usepackage{geometry}
\ifcsname bmlCrop\endcsname
\usepackage{crop}
\fi

% collapsible titled frame
\usepackage{amssymb}
\usepackage{framed}
\iflatexml
  % if compiling via LaTeXML, emit <DETAILS><SUMMARY>... tags
  \newenvironment{foldedframe}[1][]{%
    \<DETAILS>\<SUMMARY>\textbf{#1}\</SUMMARY>}%
  {\</DETAILS>}
\else
  % if compiling to PDF, emit {titled-frame} using the framed package
  \newenvironment{foldedframe}[1][]{\bgroup\colorlet{TFFrameColor}{SpringGreen}
  \colorlet{TFTitleColor}{black}\begin{titled-frame}{#1}}{\end{titled-frame}\egroup}
\fi

\usepackage{enumitem}

\usepackage{tabularx}
\renewcommand\tabularxcolumn[1]{m{#1}}% for vertical centering text in X column

\usepackage{fancyhdr}
\usepackage{geometry}
\setlength{\parindent}{0pt}
\setlength{\parskip}{0.3em}

\usepackage[all]{xy}

\usepackage{tikz}
\usetikzlibrary{cd}
\usetikzlibrary{ducks}
\newcommand\tikzname{Ti\emph{k}Z}

\usepackage{animate}
\bmlImageEnvironment{animateinline}

% for code with syntax highlighting
\usepackage{listings}
\lstset{basicstyle={\small\ttfamily},%
  keywordstyle={\color{blue}\bfseries},%
  keywordstyle=[2]{\color{MidnightBlue}\bfseries},%
  stringstyle={\color{red}},%
  commentstyle={\color{OliveGreen}},%
  frame={single},frameround={tttt},rulecolor={\color{SpringGreen}},%
  upquote=true%
}

\lstdefinelanguage{LaTeX}[LaTeX]{TeX}{%
  moretexcs={chapter,text,RequirePackage,includegraphics,bf}%
}

\lstdefinestyle{latexml}{language=LaTeX,
  texcsstyle=*{\color{blue}\bfseries},
  texcsstyle=*[2]{\color{MidnightBlue}\bfseries},
  moretexcs=[2]{LaTeXML,iflatexml,lxAddClass,lxWithClass,lxBeginTableHead,lxEndTableHead,
    lxFcn,lxID,lxPunct,lxContextTOC,lxNavbar,lxHeader,lxFooter,ldsHTML,bmlRawHTML,
    bmlImageEnvironment,bmlDescription,bmlHTMLEnvironment,bmlHTMLInlineEnvironment,bmlDisableMathJax,<,}
}

\lstdefinestyle{bookml}{language=[LaTeX]TeX,
  texcsstyle=*{\color{blue}\bfseries},
  texcsstyle=*[2]{\color{MidnightBlue}\bfseries},
  moretexcs={xymatrix,ltxinline,href,RequirePackage},
  moretexcs=[2]{LaTeXML,iflatexml,lxAddClass,lxWithClass,lxBeginTableHead,lxEndTableHead,lxFcn,lxID,lxPunct,lxContextTOC,lxNavbar,lxHeader,lxFooter,bmlImageEnvironment,bmlHTMLEnvironment,bmlRawHTML,bmlDescription,bmlPlusClass,bmlHTMLInlineEnvironment,bmlAltFormat}
}

\lstdefinestyle{yaml}{
  basicstyle={\small\ttfamily\color{blue}},
  comment=[l]{:},
  commentstyle={\color{blue!50!black}},
}

\def\ltxinline{\lstinline[style=bookml,frame=none]}
\def\htmlinline{\lstinline[language=html,frame=none]}
\def\cmdinline{\lstinline[language=bash,frame=none]}
\def\makeinline{\lstinline[language=make,frame=none]}
\def\yamlinline{\lstinline[style=yaml,frame=none]}

\title{The Leeds BookML guide}
\author{Vincenzo Mantova}

\date{21\textsuperscript{st} January 2026}

\begin{document}

\begin{lxFooter}
  Copyright \copyright{} 2021--26 Vincenzo Mantova, University of Leeds.
\end{lxFooter}
\fancyfoot[L]{Copyright \copyright{} 2021--26 Vincenzo Mantova, University of Leeds.}
\fancyfoot[C]{}
\fancyfoot[R]{\thepage}
\pagestyle{fancy}

\maketitle

This is a short manual on how to install and run BookML at the University of Leeds, accompanied by some general tips about improving the accessibility of your documents. For further instructions about BookML, including explanations about \tikzname, alternative text, videos, and so on, please consult the \href{https://vlmantova.github.io/bookml/}{BookML manual}.

\tableofcontents

\section{Installation}
\subsection{Prerequisites}

\begin{foldedframe}[Full list of prerequisites]
  \begin{itemize}
    \item \LaTeXML{} (minimum 0.8.7, recommended 0.8.8 or later)
    \item Any image handling: the Perl module \texttt{Image::Magick}
    \item Support for converting EPS, PDF images to SVG: Ghostscript, \texttt{dvisvgm} for EPS; \texttt{mutool} for PDF (you can also configure BookML to use \texttt{dvisvgm} instead of \texttt{mutool}, but it is less reliable)
    \item BookML images: Ghostscript, \texttt{latexmk}, \texttt{preview.sty}, \texttt{dvisvgm} (minimum 1.6, recommended 2.7 or later)
    \item Automatic PDF, \HTML, and zip creation: GNU Make, \texttt{latexmk}, \texttt{zip}, optionally \texttt{texfot}
  \end{itemize}
\end{foldedframe}

The packages \texttt{latexmk}, \texttt{preview.sty}, \texttt{dvisvgm} and \texttt{texfot} are distributed by MiK\TeX{}, \TeX{} Live, and virtually all Linux distributions.

For the rest of the software, follow the instructions below.

\begin{foldedframe}[Overleaf]
  \begin{itemize}
    \item \textbf{First time only.} Open your Overleaf \href{https://www.overleaf.com/user/settings}{account settings} and \href{https://docs.overleaf.com/integrations-and-add-ons/git-integration-and-github-synchronization/github-synchronization#linking-your-overleaf-account-to-your-github-account}{link your GitHub account} from the `GitHub Sync' entry.
    \item Open an existing or new Overleaf project and start \href{https://docs.overleaf.com/integrations-and-add-ons/git-integration-and-github-synchronization/github-synchronization#creating-a-new-github-repository-from-an-overleaf-project}{synchronizing it with a new GitHub repository}. The process is smoother when \textbf{the project name has no spaces}. \textbf{Note:} I recommend you immediately visit the new repository and click the `Watch' icon (next to `Edit Pins', `Fork', `Star').
    \item Finally, add the file \texttt{.github/workflows/bookml.yaml} to the project with the following content:
      \begin{lstlisting}[style=yaml,name=bookml.yaml]
on: push
jobs:
  build:
    runs-on: ubuntu-latest
    permissions:
      contents: write
    steps:
      - name: Compile with BookML
        uses: vlmantova/bookml-action@v1
      \end{lstlisting}
      Now synchronize the project with GitHub. \textbf{First time only:} GitHub will ask you to grant permissions to Overleaf to run workflows. Say yes!
    \item GitHub should start compiling all \texttt{.tex} files at the top folder of your Overleaf project that contain the string \ltxinline|\documentclass| (even if commented out!). The work is done by the BookML Docker image, which at the moment contains a full copy of \TeX{} Live 2021.
    \item If you `watched' the repository as recommended in the previous steps, you should receive an email in 2--3 minutes with the download links of all the outputs compiled by BookML and any relevant error messages. If you do not watch the repository, you must visit the repository page yourself and look for new entries under `Releases' (below `About') or open the latest workflow run under the `Action' tab.
    \item Every time you want to compile a new version of your Overleaf project, just push the new changes to GitHub.
  \end{itemize}
\end{foldedframe}

\begin{foldedframe}[GitHub]
  \begin{itemize}
    \item Simply add the file \texttt{.github/workflows/bookml.yaml} to your repositories with the following content:
      \begin{lstlisting}[style=yaml,name=bookml.yaml]
on: push
jobs:
  build:
    runs-on: ubuntu-latest
    permissions:
      contents: write
    steps:
      - name: Compile with BookML
        uses: vlmantova/bookml-action@v1
      \end{lstlisting}
      I also recommend that you watch the repository (this is usually automatic when you create a repository; see the `Watch' icon on the repository page, next to `Edit Pins', `Fork', `Star').
    \item GitHub should start compiling all \texttt{.tex} files at the top folder of your repository that contain the string \ltxinline|\documentclass| (even if commented out!). The work is done by the BookML Docker image, which at the moment contains a full copy of \TeX{} Live 2021.
    \item If you are watching the repository as recommended in the previous steps, you should receive an email in 2--3 minutes with the download links of all the outputs compiled by BookML and any relevant error messages. If you do not watch the repository, you must visit the repository page yourself and look for new entries under `Releases' (below `About') or open the latest workflow run under the `Action' tab.
    \item Every push will start a new compile. If you push frequently, consider tweaking the action to run only when pushing to \href{https://docs.github.com/en/actions/reference/workflows-and-actions/events-that-trigger-workflows#running-your-workflow-only-when-a-push-to-specific-branches-occurs}{specific branches} or \href{https://docs.github.com/en/actions/reference/workflows-and-actions/events-that-trigger-workflows#running-your-workflow-only-when-a-push-of-specific-tags-occurs}{tags}.
  \end{itemize}
\end{foldedframe}

\begin{foldedframe}[macOS (MacPorts)]
  \begin{itemize}
    \item Open the Terminal app.
    \item Run \cmdinline|xcode-select --install| to get the Command Line Developer Tools.
    \item Install MacPorts as per \href{https://www.macports.org/install.php}{its official instructions} \textbf{from point 2} (no need for full Xcode!).
    \item If your \LaTeX{} was installed via Mac\TeX{}, run:
          \begin{lstlisting}
sudo port install gmake LaTeXML +mactex
        \end{lstlisting}
        For the Ghostscript and mutool prerequisites, depending on your specific version of Mac\TeX{}, you may need to install the \href{https://tug.org/mactex/morepackages.html}{Ghostscript-Extras package}.

        Otherwise
        \begin{lstlisting}[language=bash]
sudo port install gmake LaTeXML
        \end{lstlisting}

      For the remaining prerequisites (such as \texttt{dvisvgm}), add them as you would add any other \TeX{} Live package (for instance using the \TeX{} Live Utility).

      Ghostscript-Extras will include mutool, but if it is still not available on your system and you use PDF images, run
      \begin{lstlisting}[language=bash]
sudo port install mupdf
      \end{lstlisting}
    \item If, \textbf{and only if}, your \LaTeX{} was installed via MacPorts, you can add the optional packages via:
          \begin{lstlisting}[language=bash]
# for automatic PDF, html, zip creation
sudo port install dvisvgm latexmk
# for BookML images (preview.sty)
sudo port install texlive-latex-extra
# for texfot (to reduce latex output during PDF creation)
sudo port install texlive-bin-extra
# for PDF to SVG conversion (normally already installed at this point)
sudo port install mupdf
          \end{lstlisting}
    \item To upgrade:
          \begin{lstlisting}
sudo port selfupdate
sudo port upgrade outdated
        \end{lstlisting}
    \item The remaining packages (e.g. GNU Make) are already available.
  \end{itemize}
\end{foldedframe}

\begin{foldedframe}[macOS (Homebrew) -- AVOID PLEASE]
  The Homebrew version has intrinsic packaging issues and \textbf{should be avoided}. However, if you really insist:
  \begin{lstlisting}
brew install latexml make
  \end{lstlisting}
  Functionality related to images will likely be broken.
\end{foldedframe}

\begin{foldedframe}[Linux Debian-based (Ubuntu, Debian, Mint, etc)]
  Download the package for the latest Ubuntu releases at \url{https://launchpad.net/ubuntu/+source/latexml}. Install \texttt{ghostscript}, \texttt{make}, \texttt{latexmk}, \texttt{dvisvgm}, \texttt{preview-latex-style}, \texttt{texlive-extra-utils} (for \texttt{texfot}), \texttt{zip}, according to your needs.
  \begin{lstlisting}[language=bash]
sudo dpkg -i latexml_0.8.8-1_all.deb
sudo apt -f install
sudo apt install ghostscript make latexmk dvisvgm preview-latex-style texlive-extra-utils zip
# for PDF to SVG conversion (normally already installed at this point)
sudo apt install mupdf-tools
  \end{lstlisting}
\end{foldedframe}

\begin{foldedframe}[Linux RPM-based (Red Hat, CentOS, AlmaLinux, etc)]
  Not figured out yet!
\end{foldedframe}

\begin{foldedframe}[Linux School PC (presumably only desktop connected via cable)]
  Run the following each time you open a new terminal:
  \begin{lstlisting}
module load latexml
module load texlive
    \end{lstlisting}
  If it does not work, run the following command and try again:
  \begin{lstlisting}
module use /apps/linsw1/modulefiles/7/
    \end{lstlisting}
  Everything else should already be available. Please contact IT if you are still on an old version of \LaTeXML{} (that is, before 0.8.8).
\end{foldedframe}

\begin{foldedframe}[Windows (AppsAnywhere --- BROKEN)]
  \textcolor{red}{\textbf{Broken as of January 2026.}} If you already have \LaTeXML{} installed via AppsAnywhere, you can work as usual. New installations and reinstallations are very likely to fail. While we search for the cause of the problem, please use the `without admin' instructions below.
  \begin{itemize}
    \item Make sure to have \href{https://it.leeds.ac.uk/it?id=kb_article&sysparm_article=KB0014827}{AppsAnywhere} installed.
    \item Install Ghostscript, ImageMagick, MiK\TeX{}, StrawberryPerl.
    \item Open a terminal and install mutool:
      \begin{lstlisting}
winget install mutool
      \end{lstlisting}
    \item Open StrawberryPerl and run:
      \begin{lstlisting}
cpanm --build-args=CC=c++ --notest --verbose Image::Magick
cpanm --notest --verbose LaTeXML
      \end{lstlisting}-
  \end{itemize}
  Everything else should then be available. Re-do all of the above to update. You may have to reinstall the apps every one or two months (open the CloudPaging Player to check their status).
\end{foldedframe}

\begin{foldedframe}[Windows (with admin rights)]
  For University laptops: you can gain admin rights by right-clicking an the installer and choosing ``Request Run as Administrator''.
  \begin{itemize}
    \item Install \href{https://strawberryperl.com/}{StrawberryPerl} \texttt{64bit} version.
    \item Install \href{https://imagemagick.org/script/download.php#windows}{ImageMagick} \texttt{x64-dll}. During installation, enable `Install development headers and libraries for C and C++':
    \begin{center}
      % change width on PDF only
      \iflatexml
        \includegraphics{imagemagick-installer-screenshot.png}
      \else
        \includegraphics[width=6cm]{imagemagick-installer-screenshot.png}
      \fi
    \end{center}
    Be very careful \textbf{not} to choose 32bit, portable, or static variants.
    \item Install \href{https://www.ghostscript.com/download/gsdnld.html}{Ghostscript} 64bit.
    \item Open a terminal and install mutool:
      \begin{lstlisting}
winget install --scope=machine mutool
      \end{lstlisting}
    \item In StrawberryPerl, run
          \begin{lstlisting}
cpanm --build-args=CC=c++ --notest --verbose Image::Magick
cpanm --notest --verbose LaTeXML
    \end{lstlisting}
  \end{itemize}
  Everything else should then be available. Re-do all of the above to update.
\end{foldedframe}

\begin{foldedframe}[Windows (without admin rights)]
  \textcolor{red}{\textbf{Remove Perl installed from AppsAnywhere.}} If you are here because the AppsAnywhere method is failing, please remove Perl from the CloudPaging Player first (open the CloudPaging Player, select StrawberryPerl, and click `Remove').

  Install the \href{https://scoop.sh/}{Scoop package manager} (no admin required). Then run:
  \begin{lstlisting}
scoop install perl
  \end{lstlisting}
  Now launch `Edit environment variables for your account', double click or `PATH', and add \verb|%USERPROFILE%\scoop\apps\perl\current\c\bin|. Close and reopen the terminal and check whether you can run \cmdinline|gmake| now. If you succeed, run:
  \begin{lstlisting}
scoop install imagemagick
scoop install ghostscript
cpanm --build-args=CC=c++ --notest --verbose Image::Magick
cpanm --notest --verbose LaTeXML
  \end{lstlisting}

  For PDF to SVG conversion, run:
  \begin{lstlisting}
winget install mutool
  \end{lstlisting}

  Everything else should then be available. Use \cmdinline|scoop update --all| to update.
\end{foldedframe}

\begin{foldedframe}[Docker/Podman/etc]
  You can also run BookML using Docker. The default image will compile automatically all the content of the \texttt{/source} folder. For instance:
  \begin{lstlisting}
docker run --rm -i -t -v.:/source ghcr.io/vlmantova/bookml
  \end{lstlisting}
  Please note that the Docker image includes a full copy of TeX Live 2021.
\end{foldedframe}

\subsection{BookML}

Unzip the \texttt{\href{https://github.com/vlmantova/bookml/releases/latest/download/template.zip}{template.zip} }file from the \href{https://github.com/vlmantova/bookml/releases/latest}{latest BookML release}. Open a terminal in the folder containing \texttt{Makefile}, \texttt{template.tex}, and run
\begin{lstlisting}[language=bash]
make detect  # for Linux
gmake detect # for macOS, Windows
\end{lstlisting}

\begin{foldedframe}[How do I run commands in the terminal?]
  You simply type them and press \texttt{<ENTER>}. To open the terminal in a specific folder:

  \begin{foldedframe}[Windows]
    Right click on a folder and select ``Open in Windows Terminal''. If the option is missing, try (re)installing it via
    \begin{lstlisting}
winget install --id Microsoft.WindowsTerminal -e
    \end{lstlisting}
  \end{foldedframe}
  \begin{foldedframe}[macOS]
    Right-click on a folder and select ``New Terminal at Folder''.
  \end{foldedframe}
  \begin{foldedframe}[Linux]
    Many file browsers have an option ``Open in Terminal'' when you right-click on a folder.
  \end{foldedframe}
\end{foldedframe}

Use \cmdinline|gmake detect| on Windows. You should get something like the following:

\begin{tabularx}{\textwidth}[h]{>{\color{MidnightBlue}\tt}r@{\texttt{\ }}>{\tt}X}
  Main files:    & \textcolor{OliveGreen}{template.tex}                                          \\
  BookML:        & \textcolor{OliveGreen}{\BookMLversion{} OK}                                   \\
  GNU Make:      & \textcolor{OliveGreen}{4.4.1 OK}                                              \\
  TeX:           & \textcolor{OliveGreen}{MiKTeX 25.4 OK}                                        \\
  perl:          & \textcolor{OliveGreen}{v5.42.0 OK}                                            \\
  LaTeXML:       & \textcolor{OliveGreen}{0.8.8 OK}                                              \\
  Image::Magick: & \textcolor{red}{NOT FOUND} (required for any image handling)                  \\
  Ghostscript:   & \textcolor{OliveGreen}{9.25 OK} (required for BookML images, EPS to SVG; may be required for PDF to SVG) \\
  mutool:        & \textcolor{OliveGreen}{1.23.0 OK} (required for PDF to SVG if using PDFTOSVG\_CONVERTER=mutool (current value `mutool')) \\
  dvisvgm:       & \textcolor{OliveGreen}{3.4.3 OK} (required for BookML images, EPS to SVG, PDF to SVG if using PDFTOSVG\_CONVERTER=dvisvgm (current value `mutool')) \\
  dvisvgm/libgs: & \textcolor{OliveGreen}{9.25 OK} (required for BookML images, EPS to SVG, PDF to SVG if using PDFTOSVG\_CONVERTER=dvisvgm (current value `mutool'))  \\
  latexmk:       & \textcolor{OliveGreen}{4.87 OK}                                               \\
  texfot:        & \textcolor{OliveGreen}{1.48 OK} (optional, for hiding some LaTeX messages)    \\
  preview.sty:   & \textcolor{OliveGreen}{14.0.6 OK} (required for BookML images)                \\
  zip:           & \textcolor{OliveGreen}{3.1b OK}                                               \\
  curl:          & \textcolor{OliveGreen}{8.13.0 OK} (required for updating with `make update')
\end{tabularx}

Anything missing will show a red \textcolor{red}{\tt NOT FOUND}. If the version is too old, there will be a red or a yellow prompt to upgrade.

\textbf{First try?} Run \cmdinline|make| (or \cmdinline|gmake| on macOS, Windows). After a bit, you will find a folder \texttt{template} and two zip files \texttt{template.zip}, \texttt{SCORM.template.zip}. Please open \texttt{template/index.html} and verify that it looks as you would expect. The zip files are set up for upload on Minerva.

\textbf{To update BookML}, run \lstinline|make check-for-update|, then \cmdinline|make update|, simply replace the \texttt{bookml} folder with the content of a newly downloaded \href{https://github.com/vlmantova/bookml/releases/latest/download/release.zip}{\texttt{release.zip}}.

\section{Converting your files}

\subsection{How to convert}
Once your are satisfied that the template is working, drop your own files next to \texttt{template.tex}: each file containing \ltxinline|\documentclass| will be treated the same as \texttt{template.tex}, and will be compiled to PDFs, SCORM and zip packages. Run \cmdinline|make detect| again, and check that `Main files' contains such new files.

There is now a reasonable chance that your files are already working (unless you are using \tikzname, which needs some care), but before your first try, you should truncate your files with an early \ltxinline|\end{document}|, as compilation times can be long and most errors will originate in the preamble anyway.

To compile, do the following:
\begin{itemize}
  \item Run
        \begin{lstlisting}
make
        \end{lstlisting}
        On Windows, it is \cmdinline|gmake| rather than \cmdinline|make|.

        After a bit, you will have files named like \texttt{lecturenotes.zip}, \texttt{SCORM.lecturenotes.zip}. The former works with `Upload Zip package'; the latter with `Create SCORM package'.

        To generate one particular package, for instance the one for the file \texttt{lecturenotes.tex}, run
        \begin{lstlisting}
make lecturenotes.zip
        \end{lstlisting}
        or, if you are using SCORM packages,
        \begin{lstlisting}
make SCORM.lecturenotes.zip
        \end{lstlisting}
  \item When you change a file and want to regenerate the zip files, just run \cmdinline|make| again. Only the files that need updating will be recompiled.
  \item If you start getting strange error messages from \texttt{make} itself, consider deleting the \texttt{auxdir} folder and try again.
\end{itemize}

\begin{foldedframe}[What is \texttt{make} doing? Step-by-step look under the hood]
  Each time you call \texttt{make}, it does the following.
  \begin{enumerate}[start=0]
    \item Read the file \texttt{Makefile} in the folder you are in. That file will import instructions from the \texttt{bookml} folder about what to do next.

    \item \label{find-main} Check all \texttt{.tex} files in the folder and find the ones containing \ltxinline|\documentclass|.

    {\small In \texttt{example.zip}, it finds \texttt{main.tex} and \texttt{secondfile.tex}.}

    \item \label{targets} Arrange to generate two `targets', a zip package and a SCORM package, for every such file in step~\ref{find-main}.

    {\small In the example, the targets are \texttt{main.zip}, \texttt{SCORM.main.zip}, \texttt{secondfile.zip}, \texttt{SCORM.secondfile.zip}.}

    \item \label{prereqs} Check if the targets of step~\ref{targets} exist, and if so, if any of their `prerequisites' are newer, in which case the targets must be updated. The prerequisites themselves are checked recursively to see if they also need to be updated. The prerequisite chain looks like `\texttt{zip => html => xml => pdf => tex}'.

    {\small In the example, on your first try, Make will follow the following prerequisite chain: \texttt{main.zip => main/index.html => main.xml => main.pdf => main.tex}. It will then build the prerequisites backwards until all is in place to create \texttt{main.zip}. Likewise for the other targets.}

    \item \label{pdfs} Build the PDF of each \LaTeX{} file found in step~\ref{find-main}, using \texttt{latexmk} to run \texttt{pdflatex}, \texttt{makeindex}, \texttt{bibtex} and similar as many times as necessary; record which files are \ltxinline|\input|'d and mark them as prerequisites for step~\ref{prereqs}.

    {\small In the example, it builds \texttt{main.pdf}, \texttt{secondfile.pdf}, and marks \texttt{chapter1.tex} as prerequisite of \texttt{main.pdf}. Any update to \texttt{chapter1.tex} will cause \texttt{main.pdf} to be updated next time you call \texttt{make}.}

    \item Call \LaTeXML{} to build an \XML{} file from each file in step~\ref{find-main}.

    {\small In the example, \texttt{main.xml}, \texttt{secondfile.xml}.}

    \item Try to build or update the alternative formats requested using \ltxinline|\bmlAltFormat|. For now, BookML only knows how to build PDF files from \LaTeX{} files that have the same name; if you need other formats, you need to build them yourself, or add the relevant instructions (called `recipes') in the \texttt{Makefile}.

    {\small In the example, \texttt{main-sans.pdf} and \texttt{main-sans-large.pdf} are the alternative formats requested in the preamble of \texttt{main.tex}, on top of \texttt{main.pdf} which has been built already. Since the example contains \texttt{main-sans.tex}, \texttt{main-sans-large.tex}, BookML will know what to do, and generate the alternative PDFs by repeating step~\ref{pdfs}.}

    \item Convert the \XML{} files to \HTML{}, using \texttt{latexmlpost}.

    {\small Thus generate the folders \texttt{main}, \texttt{secondfile}, each containing an \texttt{index.html}.}

    \item Zip the folder and pack the SCORM package (the latter requiring a couple more steps I will not explain), using \texttt{zip}

    {\small At last, you will get \texttt{main.zip}, \texttt{SCORM.main.zip}, \texttt{secondfile.zip}, \texttt{SCORM.secondfile.zip}.}
  \end{enumerate}
\end{foldedframe}

\subsection{Necessary adjustments}
Consult \texttt{template.tex} for the minimal requirements in the preamble (e.g.\ you \textbf{must} provide a \ltxinline|\title| command). You should create copies of \texttt{template-sans.tex}, \texttt{template-sans-large.tex} for each of your main files, at least as a baseline. Alternative versions can be customized and removed, as explained in the next subsection. We recommend offering some alternative versions as good practice.

You should also follow the key requirements below.
\begin{itemize}
  \item Use the \ltxinline|babel| package to set the document language (crucial for screen readers to work correctly). For instance:
        \begin{lstlisting}[style=latexml]
\usepackage[british]{babel}
  \end{lstlisting}
  \item Set the document metadata in the preamble (essential for proper navigation links, web page titles, SCORM package metadata, and so on).
        \begin{lstlisting}[style=latexml]
\title{LaTeXML + BookML guide}
\author{Vincenzo Mantova}
    \end{lstlisting}
  \item Ensure all \TeX{} style formatting commands (\ltxinline|\Large|, \ltxinline|\bf|, \dots) are enclosed between braces, and use \LaTeX{} alternatives such as \ltxinline|\textbf{}| when possible. If you see the wrong font in the \HTML{} output, it is likely caused by \TeX{}-style font switches that haven't gone well.
        \begin{lstlisting}[style=latexml]
{\bf some bold text}    % DO
\bf some bold text      % DON'T
\textbf{some bold text} % BEST
    \end{lstlisting}
  \item Using \tikzname{} or \Xy-pic and an old version of \LaTeXML{} (before 0.8.7), or getting misaligned pictures and very long compile times? Use the \ltxinline|bmlimage| feature (see the \href{https://vlmantova.github.io/bookml}{BookML manual} for more info).
\end{itemize}

\subsection{Alternative formats}
This is not required, but it is good to be ready to include alternative versions of the PDF as well, for instance a large print version. For instance, add the following to the preamble of \texttt{lecturenotes.tex}:
\begin{lstlisting}[style=latexml]
\usepackage{bookml/bookml} % if not already in your preamble
\bmlAltFormat{lecturenotes.LARGE.pdf}{PDF (large print)}
\end{lstlisting}
and create a corresponding \texttt{lecturenotes.LARGE.tex} that
\begin{itemize}
  \item does \textbf{NOT} contain \ltxinline|\documentclass| (or it will be compiled into its own SCORM and zip packages);
  \item configures e.g.\ different fonts and margins, then call \ltxinline|\input{lecturenotes.tex}|.
\end{itemize}
BookML will automatically compile and include \texttt{lecturenotes.LARGE.pdf} in your final outputs.

Consult \texttt{template.tex}, \texttt{template-sans.tex}, \texttt{template-sans-large.tex} for some simple techniques.

If you instead want \textbf{distinct} SCORM and zip packages, for instance compile a problem sheet both with and without solutions, explore \href{https://github.com/vlmantova/bookml/releases/latest/download/example.zip}{\texttt{example.zip}} to see some possibilities.

\subsection{Other adjustments}
Please peruse the \href{https://vlmantova.github.io/bookml}{BookML manual} for several suggestions, ranging from how to write foldable environments, embed videos and animations, deal with very slow or broken \tikzname{} pictures, and so on.

\section{Uploading to Minerva}
\subsection{SCORM packages}
\begin{enumerate}
  \item On the front page of your module, under `Module content', click the $\oplus$ button where you want to insert your item.
  \item Choose `$\oplus$ Create'.
  \item Choose `SCORM package'.
  \item Choose `Upload SCORM package' and select your \texttt{SCORM.<...>.zip} file.
  \item Disable `Grade SCORM' and click `Save'.
\end{enumerate}

\textbf{Note.} Title and abstract of the file will become title and description of the Minerva entry. A better way to prepare the description is to use
  \begin{lstlisting}[style=bookml]
\usepackage{hyperref}
\hypersetup{pdfsubject={Description of this SCORM package}}
  \end{lstlisting}
You will be able to edit the Minerva details right after uploading, if necessary.

\subsection{Plain ZIP packages}
For the initial permission setup, as well as screenshots of the entire process, consult \href{https://leeds365.sharepoint.com/sites/Maths-TeachingStaffOnly/SitePages/Creating-accessible-content-and-uploading-HTML.aspx}{Chris' guide}.

Below is a summary of the day-to-day upload process, once permissions have been set up:
\begin{enumerate}
  \item On the front page of your module, under `Module content', click the $\oplus$ button where you want to insert your item.
  \item Choose `Content Collection', then `Browse Content Collection'.
  \item Browse to the folder that has been set up with the appropriate permisions.
  \item Click `Upload' and choose `Upload Zip Package'.
  \item Use `Browse Local Files' to upload your \texttt{<...>.zip} file. You \textbf{must enable} `If selected, the system automatically overwrites the existing file with the same name'.
  \item Click `Submit', the choose `index.html' as file to be presented on Minerva.
\end{enumerate}

\section{Some examples}

\begin{figure}[hb]
  \begin{center}
    \begin{bmlimage}
      \begin{tikzpicture}
        \draw (0.5,0.5) node {$2$};
        \draw (1.5,0.5) node {$+$};
        \draw (3,0.5) node {$3$};
        \draw (4.5,0.5) node {$=$};
        \draw (7,0.5) node {$5$};
        \draw (0,0) node [below] {$0$} -- (1,0) node [below] {$1$};
        \draw (1.5,0) node {$+$};
        \draw (2,0) node [below] {$0$} -- (3,0) node [below] {$1$} -- (4,0) node [below] {$2$};
        \draw (4.5,0) node {$=$};
        \draw (5,0) node [below] {$0$} -- (6,0) node [below] {$1$};
        \draw [dashed] (6,0) -- (7,0);
        \draw (7,0) node [below] {$2$} -- (8,0) node [below] {$3$} -- (9,0) node [below] {$4$};
        \foreach \x in {0,1,2,3,4,5,6,7,8,9} {
            \fill (\x,0) circle (0.05);
          }
      \end{tikzpicture}
    \end{bmlimage}
    \bmlDescription{To compute 2+3, add a copy of the well ordered set 3 after the well ordered set 2, and note that the result is isomorphic to the well ordered set 5.}
  \end{center}
  \begin{center}
    \begin{tikzpicture}
      \draw (0.5,0.5) node {$2$};
      \draw (1.5,0.5) node {$+$};
      \draw (3,0.5) node {$3$};
      \draw (4.5,0.5) node {$=$};
      \draw (7,0.5) node {$5$};
      \draw (0,0) node [below] {$0$} -- (1,0) node [below] {$1$};
      \draw (1.5,0) node {$+$};
      \draw (2,0) node [below] {$0$} -- (3,0) node [below] {$1$} -- (4,0) node [below] {$2$};
      \draw (4.5,0) node {$=$};
      \draw (5,0) node [below] {$0$} -- (6,0) node [below] {$1$};
      \draw [dashed] (6,0) -- (7,0);
      \draw (7,0) node [below] {$2$} -- (8,0) node [below] {$3$} -- (9,0) node [below] {$4$};
      \foreach \x in {0,1,2,3,4,5,6,7,8,9} {
          \fill (\x,0) circle (0.05);
        }
    \end{tikzpicture}
    \bmlDescription{To compute 2+3, add a copy of the well ordered set 3 after the well ordered set 2, and note that the result is isomorphic to the well ordered set 5.}
  \end{center}
  \caption{Ordinal sum of $2$ and $3$ generated in two ways: first via `bmlimage', then directly by \LaTeXML{}. The latter uses MathJax for the embedded formulas, resulting in improved accessibility but minor alignment issues.}
  \label{fig:tikz-example}
\end{figure}

\begin{figure}[hb]
  \begin{center}
    \begin{bmlimage}
      \begin{tikzcd}
        A \arrow[rd] \arrow[r, "\phi"] & B \\
        & C
      \end{tikzcd}
    \end{bmlimage}
    \bmlDescription{A, B, C drawn in a triangle with C under B, an arrow labelled phi from A to B and an arrow from A to C}
  \end{center}
  \begin{center}
    \begin{tikzcd}
      A \arrow[rd] \arrow[r, "\phi"] & B \\
      & C
    \end{tikzcd}
    \bmlDescription{A, B, C drawn in a triangle with C under B, an arrow labelled phi from A to B and an arrow from A to C}
  \end{center}
  \caption{Example of tikzcd diagram, again generated in two ways via `bmlimage' and directly by \LaTeXML{}.}
  \label{fig:tikzcd-example}
\end{figure}

\begin{figure}[hb]
  \begin{lstlisting}[style=latexml]
\begin{bmlimage}
  \[ \xymatrix{
        A \ar[rd] \ar^\phi[r] & B \\
                              & C } \]
\end{bmlimage}
\bmlDescription{A, B, C drawn in a triangle with C under B,
  an arrow labelled phi from A to B and an arrow from A to C}
  \end{lstlisting}
  \begin{center}
    \[ \begin{bmlimage} \xymatrix{
          A \ar[rd] \ar^\phi[r] & B \\
      & C } \end{bmlimage} \]
    \bmlDescription{A, B, C drawn in a triangle with C under B, an arrow labelled phi from A to B and an arrow from A to C}
  \end{center}
  \caption{Example of \Xy-matrix diagram processed using \texttt{bmlimage}.}
  \label{fig:xymatrix-example}
\end{figure}
\begin{figure}
  \[ \xymatrix{
      A \ar[rd] \ar^\phi[r] & B \\
      & C } \]
  \caption{Unsatisfying example of \Xy-matrix generated directly by \LaTeXML{}.}
\end{figure}

\begin{figure}[hb]
  \begin{lstlisting}[style=latexml]
% preamble
\newcommand{\youtube}[2]{\bmlRawHTML{<iframe
  src="https://www.youtube-nocookie.com/embed/#1"
  title="#2" frameborder="0" allow="picture-in-picture; web-share"
  referrerpolicy="strict-origin-when-cross-origin" allowfullscreen=""
  style="width:100\%;max-width:1920px;aspect-ratio:16/9"></iframe>}
  Watch \href{https://www.youtube.com/watch?v=#1}{#2}}
% document
\youtube{pTqBXcUPDug?si=tfkXijIyz4dgB38Z}{SCORM upload to Blackboard Learn Ultra}
  \end{lstlisting}
  \begin{quote}
    \newcommand{\youtube}[2]{\bmlRawHTML{<iframe
      src="https://www.youtube-nocookie.com/embed/#1"
      title="#2" frameborder="0" allow="picture-in-picture; web-share"
      referrerpolicy="strict-origin-when-cross-origin" allowfullscreen=""
      style="width:100\%;max-width:1920px;aspect-ratio:16/9"></iframe>}
      Watch \href{https://www.youtube.com/watch?v=#1}{#2}}
    \youtube{pTqBXcUPDug?si=tfkXijIyz4dgB38Z}{SCORM upload to Blackboard Learn Ultra}
  \end{quote}
  \label{fig:embed-stream}
  \caption{How to embed a video. Note that the \LaTeX{} special characters are preceeded by a backslash or the output may be invalid.}
\end{figure}

\begin{figure}[hb]
  \begin{lstlisting}[style=latexml]
\begin{tabularx}{\textwidth}{c|X||c}
  \lxBeginTableHead{} Header 1 & Header 2 & Header 3 \\
  \hline \lxEndTableHead{}
  Content & Content & Content \\
  More content & content & content \\
  \hline
\end{tabularx}
\caption{A table}
  \end{lstlisting}
  \begin{tabularx}{\textwidth}{c|X||c}
    \lxBeginTableHead{} Header 1 & Header 2 & Header 3 \\
    \hline \lxEndTableHead{}
    Content                      & Content  & Content  \\
    More content                 & content  & content  \\
    \hline
  \end{tabularx}
  \caption{Mark a table row as header. Read the content of \href{https://github.com/brucemiller/LaTeXML/blob/master/lib/LaTeXML/texmf/latexml.sty}{\texttt{latexml.sty}} for more table-related commands.}
  \label{fig:table}
\end{figure}

\end{document}
